\section{Introduction}
Efficient image retrieval relies on encoding high-dimensional visual features into lower-dimensional representations while preserving semantic relationships.
Hashing techniques have been widely adopted due to their ability to enable fast comparisons using Hamming distance. 
However, challenges such as semantic loss during hashing and sensitivity to data distribution limit their real-world effectiveness.

A two-stage retrieval approach can mitigate these issues by leveraging hashing for an initial fast search, followed by a refinement stage that improves ranking precision. This study explores:

\begin{itemize}
    \item 
    The strengths and weaknesses of various DSH methods under this two-stage framework.
    \item
    The impact of different ranking refinement strategies on retrieval accuracy.
    \item
    A fair and standardized evaluation of these techniques across diverse datasets. 
\end{itemize}

\subsection{Problem Statement}

With the exponential growth of image datasets, traditional image retrieval methods have become increasingly inefficient in terms of both accuracy and scalability. Conventional approaches struggle to handle the high-dimensional nature of image data, leading to slow retrieval speeds and excessive storage requirements. Hashing-based techniques offer a promising alternative by encoding images into compact binary codes, allowing for faster search operations with reduced memory consumption.

Despite the advancements in hashing-based retrieval, existing methods, particularly Deep Supervised Hashing, still face challenges. These include suboptimal hash code learning, difficulty in preserving semantic similarities, sensitivity to noise and variations in images, and computational overhead during training. Furthermore, many current models lack the ability to generalize well across diverse datasets, limiting their applicability in real-world scenarios.

Therefore, there is a need to explore and develop improved Deep Supervised Hashing techniques that enhance retrieval accuracy, efficiency, and robustness. 







